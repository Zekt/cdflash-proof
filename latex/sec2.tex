\section{The Program Transition System~\Prog}
\label{sec:Prog}

We assume that three sets are given,  
$$\i{Addr},\ \i{Data},\ \text{and}\ \i{ProgState}$$
where $\i{Addr}$ is the set of memory addresses, $\i{Data}$ the set of values of memory cells, $\i{ProgState}$ the set of all possible program states.
The exact definitions of these sets are irrelevant to the Agda proof and omitted.
We will need to read the memory cells in a program state, which is done by the function
$$ read : \i{ProgState} \to (\i{Addr} \to \i{Data}) $$
which returns the mapping from memory cell addresses to their values in a program state.

The (labelled) transition system~\Prog\ uses \i{ProgState} as its states, and its actions (labels) are defined by%
\todo{need to explain the meaning of the actions?}
\begin{align*}
%\noalign{\center{\small{State of Prog, we don't have to know the details.}}}
	\i{Action} \defeq{} &\{\, \actw_{\i{addr}, \i{data}} \mid \i{addr} \in \i{Addr}, \i{data} \in \i{Data} \,\} \\
	\cup \ &\{\, \actwc_{\i{addr}, \i{data}} \mid \i{addr} \in \i{Addr}, \i{data} \in \i{Data} \,\}\\
	\cup \ &\{ \actf, \actr, \actfc, \actrc \}
\end{align*}
Most of the time the pair of address and data in a $\actw$ or $\actwc$ action is not important, in which case we will omit the subscript.
The ternary (single-step) transition relation of~\Prog\ is
$${\ttP{}} \subseteq \i{ProgState} \times \i{Action} \times \i{ProgState}$$
Its exact definition is, again, irrelevant and omitted.
We write
$$s \ttP{a} s'$$
when state~$s$ transits to~$s'$ through action~$a$ in~\Prog.

Recall that there are two types of invariants on the states, the representation invariant \i{RI} and the crash invariant \i{CI}, which are preserved or transformed by the transitions; these properties are proved by an SMT solver, and are made assumptions in this proof.
\begin{assumption}[per-operation correctness]
For all $s$, $s' \in \i{ProgState}$, the following are assumed:
\begin{align*}
s \ttP{\awa{\actw}{\actw }} s' \conj \i{RI}(s) &\implies \i{RI}(s')\\
s \ttP{\awa{\actw}{\actf }} s' \conj \i{RI}(s) &\implies \i{RI}(s')\\
s \ttP{\awa{\actw}{\actwc}} s' \conj \i{RI}(s) &\implies \i{CI}(s')\\
s \ttP{\awa{\actw}{\actfc}} s' \conj \i{RI}(s) &\implies \i{CI}(s')\\
s \ttP{\awa{\actw}{\actr }} s' \conj \i{CI}(s) &\implies \i{RI}(s')\\
s \ttP{\awa{\actw}{\actrc}} s' \conj \i{CI}(s) &\implies \i{CI}(s')
\end{align*}
\end{assumption}

Below we will often need to talk about sequences of transitions in \Prog\ and other transition systems, and a general definition will be helpful.
\begin{definition}[fragments]\label{def:frag}
Let $T = (S, \Lambda, \ttIn{}{})$~be a labelled transition system.
A \emph{fragment} in~$T$ is a sequence of transitions
$$ s_0 \ttIn{a_1}{} s_1 \ttIn{a_2}{} \cdots \ttIn{a_n}{} s_n $$
where $s_i \in S$ and $a_j \in \Lambda$ for all $i$~and~$j$.
We sometimes omit the intermediate states and write
$$ s_0 \ttsIn{a_1a_2\ldots a_n}{} s_n $$
Note that a fragment may contain no transitions, in which case the starting and ending states are the same: 
$$ s_0 \ttsIn{\vphantom{a_n}}{} s_0 $$
\end{definition}

Our goal, crash-determinacy on~\Prog, can be stated as follows:
\begin{theorem}[crash-determinacy on \Prog] \label{theoremP}
	There are two possible situations where a recovery is needed, crash at write and crash at flush, we will prove that crash-determinacy would hold in both scenarios. Thus for all $s_0, s', s'', s''' \in \i{ProgState}$, the theorem can be stated separately in two cases:
\begin{align*}
%Case $\actwc$: \\
	         \intertext{Case $\actwc$:}
	         &  s_0 \ttsP{\i{((\actw \lor \actf)\ast)} \actf} s' \conj s' \ttsP{\i{(\actw \ast)} \actwc \i{(\actrc\ast)} \actr} s'' \\
	\implies & \i{read(s')} \equiv \i{read(s'')} \\
	         \intertext{Case $\actfc$:}
	         & s_0 \ttsP{\i{((\actw \lor \actf)\ast)} \actf} s' \conj s' \ttsP{\i{(\actw \ast)}} s'' \conj s'' \ttsP{\actfc \i{(\actrc \ast)} \actr} s'''  \\
	\implies & \i{read(s')} \equiv \i{read(s''')} \disj \i{read(s'')} \equiv \i{read(s''')} 
\end{align*}
\end{theorem}
where $s_0$ is an initial state in \Prog.
To prove the theorem, we would need to introduce two other transition systems~\Spec~and~\ProgInv.
