% !TEX root = main.tex

\section{The Program Transition System~\Prog}
\label{sec:Prog}

We assume that three sets are given,  
$$ \AgdaId{\id{Addr}},\ \AgdaId{\id{Data}},\ \text{and}\ \AgdaId{\ProgState} $$
where $\i{Addr}$ is the set of memory addresses, $\i{Data}$ the set of values of memory cells, $\ProgState$ the set of all possible program states.
The exact definitions of these sets are irrelevant to the Agda proof and omitted.
We will need to read the memory cells in a program state, which is done by the function
$$ \AgdaId{\id{read}} : \ProgState \to (\i{Addr} \to \i{Data}) $$
which returns the mapping from memory cell addresses to their values in a program state.
We also assume that there is a predicate $\AgdaId{\id{Init}^R}$ on \ProgState\ that is used to identify the new disk states.

The (labelled) transition system~\Prog\ uses \ProgState\ as its states, and its actions (labels) are defined by
\begin{align*}
%\noalign{\center{\small{State of Prog, we don't have to know the details.}}}
	\AgdaId{\i{Action}} \defeq{} &\{\, \actw_{\i{addr}, \i{data}} \mid \i{addr} \in \i{Addr}, \i{data} \in \i{Data} \,\} \\
	\cup \ &\{\, \actwc_{\i{addr}, \i{data}} \mid \i{addr} \in \i{Addr}, \i{data} \in \i{Data} \,\}\\
	\cup \ &\{ \actf, \actr, \actcp, \actes, \actfc, \actrc, \actcpc, \actesc \}
\end{align*}
Most of the time the pair of address and data in a $\actw$ or $\actwc$ action is not important, in which case we will omit the subscript.
We will refer to a transition system whose labels are drawn from \id{Action} as a \emph{disk model} --- in particular, \Prog\ is a disk model.
The ternary (single-step) transition relation of~\Prog\ is
$$ {\ttP{}} \subseteq \ProgState \times \i{Action} \times \ProgState \qquad \AgdaId{\_\sem{\_}^{\mathrm R}{\blacktriangleright}\_} $$
Its exact definition is, again, irrelevant for our purposes and omitted.
We write
$$s \ttP{a} s'$$
when state~$s$ transits to~$s'$ through action~$a$ in~\Prog.

Recall that there are two types of invariants on the states, the representation invariant \AgdaId{\i{RI}} and the crash representation invariant \AgdaId{\i{CI}}, which are preserved or transformed by the transitions; these properties are proved by an SMT solver, and are made assumptions in this proof.

\begin{assumption}[type B per-operation correctness]\label{peropb}
For all $s$, $s' \in \ProgState$, the following are assumed:
\begin{align*}
s \ttP{\awa{\actw}{\actw }} s' \conj \i{RI}(s) &\implies \i{RI}(s') & \tag*{\AgdaId{\id{RIRI}}} \\
s \ttP{\awa{\actw}{\actf }} s' \conj \i{RI}(s) &\implies \i{RI}(s')\\
s \ttP{\awa{\actw}{\actcp}} s' \conj \i{RI}(s) &\implies \i{RI}(s')\\
s \ttP{\awa{\actw}{\actes}} s' \conj \i{RI}(s) &\implies \i{RI}(s')\\
s \ttP{\awa{\actw}{\actwc}} s' \conj \i{RI}(s) &\implies \i{CI}(s') & \tag*{\AgdaId{\id{RICI}}} \\
s \ttP{\awa{\actw}{\actfc}} s' \conj \i{RI}(s) &\implies \i{CI}(s')\\
s \ttP{\awa{\actw}{\actcp}} s' \conj \i{RI}(s) &\implies \i{CI}(s')\\
s \ttP{\awa{\actw}{\actes}} s' \conj \i{RI}(s) &\implies \i{CI}(s')\\
s \ttP{\awa{\actw}{\actr }} s' \conj \i{CI}(s) &\implies \i{RI}(s') & \tag*{\AgdaId{\id{CIRI}}} \\
s \ttP{\awa{\actw}{\actrc}} s' \conj \i{CI}(s) &\implies \i{CI}(s') & \tag*{\AgdaId{\id{CICI}}}
\end{align*}
\end{assumption}

We also assume that the new disk states satisfy $\id{CI}$.

\begin{assumption}[\AgdaId{\ensuremath{\id{init}^R\text-\id{CI}}}]
\[ \id{Init}^R(s) \implies \id{CI}(s) \]
\end{assumption}

%Our goal, crash-determinacy on~\Prog, can be stated as follows:
%\begin{theorem}[crash-determinacy on \Prog] \label{theoremP}
%	There are two possible situations where a recovery is needed, crash at write and crash at flush, we will prove that crash-determinacy would hold in both scenarios. Thus for all $s_0, s', s'', s''' \in \i{ProgState}$, the theorem can be stated separately in two cases:
%\begin{align*}
%%Case $\actwc$: \\
%	         \intertext{Case $\actwc$:}
%	         &  s_0 \ttsP{\i{((\actw \lor \actf)\ast)} \actf} s' \conj s' \ttsP{\i{(\actw \ast)} \actwc \i{(\actrc\ast)} \actr} s'' \\
%	\implies & \i{read(s')} \equiv \i{read(s'')} \\
%	         \intertext{Case $\actfc$:}
%	         & s_0 \ttsP{\i{((\actw \lor \actf)\ast)} \actf} s' \conj s' \ttsP{\i{(\actw \ast)}} s'' \conj s'' \ttsP{\actfc \i{(\actrc \ast)} \actr} s'''  \\
%	\implies & \i{read(s')} \equiv \i{read(s''')} \disj \i{read(s'')} \equiv \i{read(s''')} 
%\end{align*}
%\end{theorem}
%where $s_0$ is an initial state in \Prog.
%To prove the theorem, we would need to introduce two other transition systems~\Spec~and~\ProgInv.
