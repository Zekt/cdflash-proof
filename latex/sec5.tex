% !TEX root = main.tex

\section{Simulation of~\ProgInv\ by~\Spec}
\label{sec:sim}
To establish the behavioral correctness and snapshot consistency of \ProgInv, we first define a simulation of \ProgInv\ by \Spec. The simulation relation is the disjunction of the two abstraction relations $\AgdaId{\i{AR}}$ and $\AgdaId{\i{CR}}$. The two relations are transformed or preserved by each operation in \ProgInv; these have been proved with an SMT solver, and are thus made assumptions.
\begin{assumption}[type A per-operation correctness]\label{peropa}
	For all $s, s' \in \ProgInvState$ and $t \in \SpecState$, the following are assumed:
\begin{align*}
s \ttPI{\awa{\actw}{\actw }} s' \conj \i{AR}(s, t) &\implies \exists t' \ldotp t \ttS{\awa{\actw}{\actw }} t' \conj \i{AR}(s', t') \tag*{\AgdaId{\id{ARAR}}}\\
s \ttPI{\awa{\actw}{\actf }} s' \conj \i{AR}(s, t) &\implies \exists t' \ldotp t \ttS{\awa{\actw}{\actf }} t' \conj \i{AR}(s', t')\\
s \ttPI{\awa{\actw}{\actcp}} s' \conj \i{AR}(s, t) &\implies \exists t' \ldotp t \ttS{\awa{\actw}{\actcp}} t' \conj \i{AR}(s', t')\\
s \ttPI{\awa{\actw}{\actes}} s' \conj \i{AR}(s, t) &\implies \exists t' \ldotp t \ttS{\awa{\actw}{\actes}} t' \conj \i{AR}(s', t')\\
s \ttPI{\awa{\actw}{\actwc}} s' \conj \i{AR}(s, t) &\implies \exists t' \ldotp t \ttS{\awa{\actw}{\actwc}} t' \conj \i{CR}(s', t')\tag*{\AgdaId{\id{ARCR}}}\\
s \ttPI{\awa{\actw}{\actfc}} s' \conj \i{AR}(s, t) &\implies \exists t' \ldotp t \ttS{\awa{\actw}{\actfc}} t' \conj \i{CR}(s', t')\\
s \ttPI{\awa{\actw}{\actcpc}} s' \conj \i{AR}(s, t) &\implies \exists t' \ldotp t \ttS{\awa{\actw}{\actcpc}} t' \conj \i{CR}(s', t')\\
s \ttPI{\awa{\actw}{\actesc}} s' \conj \i{AR}(s, t) &\implies \exists t' \ldotp t \ttS{\awa{\actw}{\actesc}} t' \conj \i{CR}(s', t')\\
s \ttPI{\awa{\actw}{\actr }} s' \conj \i{CR}(s, t) &\implies \exists t' \ldotp t \ttS{\awa{\actw}{\actr }} t' \conj \i{AR}(s', t')\tag*{\AgdaId{\id{CRAR}}}\\
s \ttPI{\awa{\actw}{\actrc}} s' \conj \i{CR}(s, t) &\implies \exists t' \ldotp t \ttS{\awa{\actw}{\actrc}} t' \conj \i{CR}(s', t')\tag*{\AgdaId{\id{CRCR}}}
\end{align*}
\end{assumption}
%We define simulation relation between a state in \ProgInvState\ and a state in \SpecState\ as either $\i{AR}$ or $\i{CR}$ holds between them.
\begin{definition}[simulation relation, \AgdaId{\id{SR}}]
    $$\i{SR(s, t)} \defeq \i{AR(s, t)} \lor \i{CR(s, t)}$$
\end{definition}

From \cref{peropa} we can show that \i{SR} is indeed a simulation relation. 
\begin{theorem}[\AgdaId{\id{simSR}}]\label{simsr}
    %    If $\i{SR}$ holds between a pair of states $s \in \ProgInvState$ and $t \in \SpecState$, and~$s'~\in~\ProgInvState$ is a state after a sequence of actions performed on~$s$, there always exists a state  $t' \in \SpecState$ such that $\i{SR}$ between~$t$~and~$t'$ holds, while $t'$ is a state after the same sequence of actions performed on $t$.
    \begin{align*}
    &\forall s, s' \in \ProgInvState, t \in \SpecState, \i{a} \in \i{Action} \ldotp\\
    &\qquad \qquad s \ttPI{\i{a}} s'  \conj \i{SR(s, t)} \implies \exists t' \ldotp t \ttS{\i{a}} t' \conj \i{SR(s', t')} \\
    \end{align*}
\end{theorem}

%\begin{corollary}[\AgdaId{\id{runSimSR}}]\label{lemma-1}
%%    If $\i{SR}$ holds between a pair of states $s \in \ProgInvState$ and $t \in \SpecState$, and~$s'~\in~\ProgInvState$ is a state after a sequence of actions performed on~$s$, there always exists a state  $t' \in \SpecState$ such that $\i{SR}$ between~$t$~and~$t'$ holds, while $t'$ is a state after the same sequence of actions performed on $t$.
%	\begin{align*}
%		&\forall s, s' \in \ProgInvState, t \in \SpecState, \i{tr} \in \i{Trace} \ldotp\\
%		&\qquad \qquad s \ttsPI{\i{tr}} s'  \conj \i{SR(s, t)} \implies \exists t' \ldotp t \ttsS{\i{tr}} t' \conj \i{SR(s', t')} \\
%	\end{align*}
%\end{corollary}


%	\begin{onehalfspacing}
%\begin{proof}
%	\todo{elaborate why there is always a next state.}
%	There exists $t' \in SpecState$, such that $t \ttsS{\i{ef}} t'$, and by applying \cref{PerOpCorrect} repeatedly, we know that~$\i{s \ttsPI{\i{ef}} s'}$, $\i{t \ttsS{\i{ef}} t'}$ and~$\i{AR(s, t)} \lor \i{CR(s, t)}$ implies either~$\i{AR(s', t')}$ or~$\i{CR(s', t')}$, as sketched in \cref{fig:sketch1}.
%\end{proof}
%	\end{onehalfspacing}
	\begin{figure}[h] \centering
\begin{tikzpicture}
  \matrix (m) [matrix of math nodes,row sep=3em,column sep=4em,minimum width=2em]
  {
	  s & \vphantom{s}\smash{s'} \\
	 t & \vphantom{t}\smash{\exists t'} \\};
  \path[-stealth]
	(m-1-1) edge [dashed, -] node [left] {$\i{SR}$} (m-2-1)
			edge [->] node [midway, above] {$ \i{a} $} node [at end, below=-0.1em] {\scriptsize{\ProgInv}}  (m-1-2)
	(m-2-1) edge [->] node [midway, above] {$ \i{a} $} node [at end, below=-0.1em] {\scriptsize{\Spec}} (m-2-2)
	(m-1-2) edge [dashed, -] node [right] {$\i{SR}$} (m-2-2);
\end{tikzpicture}
		\caption{Illustration of \cref{simsr}}
\label{fig:sketch1}
	\end{figure}

Also proved with an SMT solver, for every pair of \i{s \in \ProgInvState} and \i{t \in \SpecState} that satisfy $AR(s,t)$, \emph{observational equivalence} holds, that is, the content of~$s$ is equivalent to the content of the volatile part of~$t$.
\begin{assumption}[\AgdaId{\id{AR{\Rightarrow}ObsEquiv}}]\label{ObsEquiv}
    $$
    \forall \i{s} \in \ProgInvState,t \in \SpecState \ldotp
    \i{AR(s, t)} \implies \i{read(s)} \equiv \i{volatile(t)}
    $$
\end{assumption}

\section{Behavioral correctness\\ and snapshot consistency of \ProgInv}

To state the behavioral correctness more easily, we give some auxiliary definitions. We define two multi-recovery fragments with the same trace, one in \Prog\ and one in~\Spec, to be $\AgdaId{\i{Conformant}}$ if every pair of corresponding normal states are observationally equivalent.
Since a multi-recovery fragment consists of multiple one-recovery fragments, we define $\AgdaId{\i{Conformant{\text-}1R}}$ to describe conformance particularly between two one-recovery fragments, and $\AgdaId{\i{Conformant{\text-}all}}$ to describe conformance between two fragments where every action in the trace is successful (i.e. without crashes).\\


\begin{definition}[\i{Conformant{\text-}all}]
    For two fragments $\i{frP} = s \ttsPI{\i{tr}} s'$ and $\i{frS} = t \ttsS{\i{tr}} t'$, $\i{Conformant{\text-}all(\i{frP}, \i{frS})}$ holds if
    \begin{itemize}
        \item $\i{tr}$ is an empty trace,
    \end{itemize}
    or when $\i{tr}$ is nonempty and can be written as $\i{tr'}, a$ where $a$ is the last action, both of the following hold,
    \begin{itemize}
        \item $\i{frP'} = s \ttsPI{\i{tr'}} s''$ and $\i{frS'} = t \ttsS{\i{tr'}} t''$ satisfy $\i{Conformant{\text-}all}(\i{frP'}, \i{frS'})$,
        \item $s'$ and $t'$ are observationally equivalent.
    \end{itemize}
\end{definition}


\begin{definition}[\i{Conformant{\text-}1R}]
    For two one-recovery fragments $\i{frP} = s \ttsPI{\i{tr}} s'$ and $\i{frS} = t \ttsS{\i{tr}} t'$, $\i{Conformant\text{-}1R}(\i{frP},\i{frS})$ holds if one of the following holds.
    \begin{itemize}
        \item In the case of\\ $\id{frP} = s_1 \ttsPI{a_1,\, \ldots,\, a_{k-1}} s_2 \ttsPI{\actf,\, b_1,\, \ldots,\, b_\ell} s_3 \ttsPI{\actwc,\, (\actrc)^m,\, \actr} s_4$, and \\
        $\id{frS} = t_1 \ttsS{a_1,\, \ldots,\, a_{k-1}} t_2 \ttsS{\actf,\, b_1,\, \ldots,\, b_\ell} t_3 \ttsS{\actwc,\, (\actrc)^m,\, \actr} t_4$,\\
        $\i{Conformant\text{-}all}$ holds for the sub-fragment from $s_1$ to $s_3$ and the sub-fragment from $t_1$ to $t_3$, and $s_4$ is observationally equivalent to $t_4$.
        \item In the case of\\ $\id{frP} = s_1 \ttsPI{a_1,\, \ldots,\, a_{k-1}} s_2 \ttsPI{\actf,\, b_1,\, \ldots,\, b_\ell} s_3 \ttsPI{\actfc,\, (\actrc)^m,\, \actr} s_4$, and \\
        $\id{frS} = t_1 \ttsS{a_1,\, \ldots,\, a_{k-1}} t_2 \ttsS{\actf,\, b_1,\, \ldots,\, b_\ell} t_3 \ttsS{\actfc,\, (\actrc)^m,\, \actr} t_4$,\\
        $\i{Conformant\text{-}all}$ holds for the sub-fragment from $s_1$ to $s_3$ and the sub-fragment from $t_1$ to $t_3$, and $s_4$ is observationally equivalent to $t_4$.
        \item  In the case of\\ $\id{frP} = s_2 \ttsPI{b_1,\, \ldots,\, b_\ell} s_3 \ttsPI{\actwc,\, (\actrc)^m,\, \actr} s_4$, and \\
        $\id{frS} = t_2 \ttsS{b_1,\, \ldots,\, b_\ell} t_3 \ttsS{\actwc,\, (\actrc)^m,\, \actr} t_4$.\\
        $\i{Conformant\text{-}all}$ holds for the sub-fragment from $s_2$ to $s_3$ and the sub-fragment from $t_2$ to $t_3$, and $s_4$ is observationally equivalent to $t_4$,
        \item In the case of\\ $\id{frP} = s_2 \ttsPI{b_1,\, \ldots,\, b_\ell} s_3 \ttsPI{\actfc,\, (\actrc)^m,\, \actr} s_4$, and \\
        $\id{frS} = t_2 \ttsS{b_1,\, \ldots,\, b_\ell} t_3 \ttsS{\actfc,\, (\actrc)^m,\, \actr} t_4$,\\
        $\i{Conformant\text{-}all}$ holds for the sub-fragment from $s_2$ to $s_3$ and the sub-fragment from $t_2$ to $t_3$, and $s_4$ is observationally equivalent to $t_4$.
    \end{itemize}
%    which can be split into two traces \i{tr_1} and \i{tr_2} where \i{tr_1} is the trace before the first crash ($\actwc$ or $\actfc$), the fragments can alternatively be written as $\i{frP} = s \ttsPI{\i{tr_1}} s'' \ttsPI{\i{tr_2}}  s'$ and $\i{frS} = t \ttsS{\i{tr_1}} t'' \ttsS{\i{tr_2}}  t'$, $\i{Conformant{\text-}1R(\i{frP}, \i{frS})}$ holds if
%    \begin{itemize}
%        \item $\i{Conformant{\text-}1R( s \ttsPI{\i{tr_1}} s'', t \ttsS{\i{tr_1}} t'')}$
%        \item $s'$ and $t'$ are observationally equivalent.\\
%    \end{itemize}

\end{definition}

\begin{definition}[\i{Conformant}]
    For two multi-recovery fragments $\i{frP} = s \ttsPI{\i{tr}} s'$ and $\i{frS} = t \ttsS{\i{tr}} t'$, \i{Conformant} is inductively defined as follows:
    \begin{itemize}
        \item In the case of $\i{frP} = s \ttsPI{(\actrc)^m,\actr} s'$ and $\i{frS} = t \ttsS{(\actrc)^m,\actr} t'$, $s'$ is observationally equivalent to $t'$,
        or
        \item in the case of $\i{frP} = s \ttsPI{\i{tr}} s'' \ttsPI{\i{tr'}} s'$ and $\i{frS} = t \ttsS{(\i{tr}} t'' \ttsS{\i{tr'}} t'$, $s'$ where $\i{tr}$ is multi-recovery and $\i{tr'}$ is one-recovery, all of the following hold:
        \begin{itemize}
            \item The sub-fragment from $s$ to $s''$ and the sub-fragment from $t$ to $t''$ satisfy \i{Conformant}.
            \item The sub-fragment from $s''$ to $s'$ and the sub-fragment fron $t''$ to $t'$ satisfy \i{Conformant\text{-}1R}.
            \item $s''$ and $t''$ are observationally equivalent.
        \end{itemize} 
    \end{itemize}
\end{definition}

With $\i{Conformant}$ defined, behavioral correctness can then be stated as follows.
Note that our goal is to establish the behavioral correctness and snapshot consistency of the kind of multi-recovery fragments defined in the main text, whose traces are of the form $$(r^c)^m, r, \i{tr}_1,...,\i{tr}_n,\i{tr'}$$ where $\i{tr}_1,...,\i{tr}_n$ are one-recovery fragments, and $\i{tr'}$ is a trailing fragment without any crashes.
\begin{theorem}[behavioral correctness, $\AgdaId{\i{BC}}$]\label{bc}
For all fragments $\i{frP_1} = s \ttsPI{\i{tr}} s'$ and $\i{frP_2} = s' \ttsPI{\i{tr'}} s''$ where $\i{tr}$ is a multi-recovery trace and $\i{tr'}$ is a trace consists of only successful regular and snapshot operations, there exist two corresponding fragments $\i{frS_1} = t \ttsS{\i{tr}} t'$ and $\i{frS_2} = t' \ttsS{\i{tr'}} t''$ such that $\i{frP_1}$ and $\i{frS_1}$ satisfy $\i{Conformant}$, while $\i{frP_2}$ and $\i{frS_2}$ satisfy $\i{Conformant{\text-}all}$. 
\end{theorem}

\Cref{bc} can be established with \cref{simsr}, as described in the main text and proved in Agda.
Finally, we can show the snapshot consistency of~\Prog\ with the help of \cref{bc}, the details of which has been presented both in the main text and Agda.
\begin{theorem}[snapshot consistency of \ProgInv, $\AgdaId{\i{Prog.SC}}$]
    For every fragment in \ProgInv\ of the form $s_0 \ttsPI{tr} s_1 \ttsPI{tr'} s_2$, where $\id{Init^P}(s_0)$ holds, $\id{tr}$ is multi-recovery, and $\id{tr}'$~is one-recovery, the sub-fragment $s_1 \ttsPI{\id{tr'}} s_2$ is snapshot-consistent under the equivalence relation $\id{ER}(s, s') \defeq \id{read}(s) = \id{read}(s')$.
    
\end{theorem} 
