\section{Simulation of~\ProgInv\ by~\Spec}
\label{sec:sim}
Recall that to establish \cref{theoremP}, we first have to define simulation between \Spec\ and \Prog, then derive crash-determinacy on \Prog\ from it. Two types of simulation relations, the abstract relation $\i{AR}$ and crash relation $\i{CR}$, which are transformed or preserved on states that satisfies certain invariants(i.e. states in \ProgInv), have been proved with a SMT solver, thus are made assumptions.
\begin{assumption}[Per Operation Correctness]\label{PerOpCorrect}
	For all $s, s_0 \in \i{ProgState^\dagger}$, $t, t_0 \in \i{SpecState}$, the following are assumed:
\begin{align*}
s \ttPI{\awa{\actw}{\actw }} s' \conj t \ttS{\awa{\actw}{\actw }} t' \conj \i{AR}(s, t) &\implies \i{AR}(s', t')\\
s \ttPI{\awa{\actw}{\actf }} s' \conj t \ttS{\awa{\actw}{\actf }} t' \conj \i{AR}(s, t) &\implies \i{AR}(s', t')\\
s \ttPI{\awa{\actw}{\actwc}} s' \conj t \ttS{\awa{\actw}{\actwc}} t' \conj \i{AR}(s, t) &\implies \i{CR}(s', t')\\
s \ttPI{\awa{\actw}{\actfc}} s' \conj t \ttS{\awa{\actw}{\actfc}} t' \conj \i{AR}(s, t) &\implies \i{CR}(s', t')\\
s \ttPI{\awa{\actw}{\actr }} s' \conj t \ttS{\awa{\actw}{\actr }} t' \conj \i{CR}(s, t) &\implies \i{AR}(s', t')\\
s \ttPI{\awa{\actw}{\actrc}} s' \conj t \ttS{\awa{\actw}{\actrc}} t' \conj \i{CR}(s, t) &\implies \i{CR}(s', t')
\end{align*}
\end{assumption}
As well proved with a solver, for a pair of \i{s \in ProgState^\dagger} and \i{t \in SpecState} that satisfies abstract relation, the content of $s$ is the same with the content of the volatile part of $t$:
\begin{assumption}[Observational Equivalence]\label{ObsEquiv}
	\begin{multline*}
	 \forall \i{s} \in \i{ProgState^\dagger},t \in \i{SpecState} \ldotp \\
	 \i{AR(s, t)} \implies \i{read(s)} \equiv \i{volatile(t)} \\
	\end{multline*}
\end{assumption}
Then we can state and prove that the simulation relation is preserved between two states in~\ProgInv\ and~\Spec\ by transitions that are index by the same fragment:
\begin{lemma}\label{lemma-1}
	\begin{align*}
		&\i{SR(s, t)} \defeq \i{AR(s, t)} \lor \i{CR(s, t)}\\
		&\forall s, s' \in \i{ProgState^\dagger}, t \in \i{SpecState}, \i{ef} \in \i{Fragment} \ldotp\\
		&\qquad \qquad s \ttsPI{\i{ef}} s'  \conj \i{SR(s, t)} \implies \exists t' \ldotp t \ttsS{\i{ef}} t' \conj \i{SR(s', t')} \\
	\end{align*}
\end{lemma}
	\begin{onehalfspacing}
\begin{proof}
	\todo{elaborate why there is always a next state.}
	There exists $t' \in SpecState$, such that $t \ttsS{\i{ef}} t'$, and by applying \cref{PerOpCorrect} repeatedly, we know that~$\i{s \ttsPI{\i{ef}} s'}$, $\i{t \ttsS{\i{ef}} t'}$ and~$\i{AR(s, t)} \lor \i{CR(s, t)}$ implies either~$\i{AR(s', t')}$ or~$\i{CR(s', t')}$, as sketched in \cref{fig:sketch1}.
\end{proof}
	\end{onehalfspacing}
	\begin{figure}[h] \centering
\begin{tikzpicture}
  \matrix (m) [matrix of math nodes,row sep=3em,column sep=4em,minimum width=2em]
  {
	  s & \vphantom{s}\smash{s'} \\
	 t & \vphantom{t}\smash{\exists t'} \\};
  \path[-stealth]
	(m-1-1) edge [dashed, -] node [left] {$\i{SR}$} (m-2-1)
			edge [->>] node [midway, above] {$ \i{ef} $} node [at end, below=-0.1em] {\scriptsize{\ProgInv}}  (m-1-2)
	(m-2-1) edge [->>] node [midway, above] {$ \i{ef} $} node [at end, below=-0.1em] {\scriptsize{\Spec}} (m-2-2)
	(m-1-2) edge [dashed, -] node [right] {$\i{SR}$} (m-2-2);
\end{tikzpicture}
		\caption{Proof sketch of \cref{lemma-1}}
\label{fig:sketch1}
	\end{figure}
Based on these assumptions and \cref{lemma-1}, we will prove \cref{theoremPI} and \cref{theoremP} in next section.